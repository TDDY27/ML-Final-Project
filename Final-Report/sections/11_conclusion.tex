\section{Conclusion}
Following are a summarize of our work.
\begin{table}[h]
    \centering
    \begin{tabular}{|c|c|c|c|c|}
    \hline
    Model & Preprocess & Other Technique & Stage1 & Stage2 \\ \hline\hline
    KNN   & TODO   & None & TODO & TODO  \\ \hline
    Linear Regression & TA \& Diff \& Ratio & None & 0.58147 & 0.5482 \\ \hline
    Logistic Regression & TODO & None & 0.571 & 0.5385 \\ \hline
    SVM & TA & Bl \& Ba \& Op & 0.57887 & 0.53349 \\ \hline
    SVM (Linear) & TA & GS & 0.57142 & 0.50980 \\ \hline
    SVM (RBF) & TA & GS & 0.57110 & 0.54656 \\ \hline
    Random Forest & TODO & None & 0.58989 & 0.5245 \\ \hline
    XGboost & TA & Op & 0.57725 & 0.54166 \\ \hline
    CATboost & DSS & Op & 0.57985 & 0.54493 \\ \hline
    \end{tabular}
    \label{tab:example_table}
\end{table}
\subsubsection*{Abbreviation}
\begin{itemize}
    \item[TA] Team Attribute
    \item[DSS] Drop std and skew 
    \item[GS] Grid Search
    \item[Bl] Blending
    \item[Ba] Bagging
    \item[Op] Optuna  
\end{itemize}

\par We suggest that you should use our Random Forest model for same season prediction and our Linear Regression model for predicting a new season. 
\par Noted that we have done Diff \& Ratio Technique on our data in the Linear Regression model. This give a inspire that using our Diff \& Ratio pre process, we can make data more linear separable (given that Linear SVM that only do Team Attribute preprocess doesn't give a as well result).
